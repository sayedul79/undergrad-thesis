\chapter{LITERATURE REVIEW}
\section{Overview of CMOS Power Amplifier}
A power amplifier employing complementary metal-oxide semiconductor (CMOS) technology is a type of electrical circuit. It is essential to many applications, including satellite communication, Wi-Fi, Bluetooth, and wireless communication networks like cellular networks.

In comparison to alternative amplifier technologies, CMOS power amplifiers provide a number of benefits, including high integration density, low power consumption, and compatibility with traditional CMOS processes. They are very appealing for usage in portable and low-power devices where cost effectiveness and power efficiency are crucial.

A power amplifier's main job is to boost an input signal's power level to a level appropriate for transmission or driving a load. This is accomplished in the case of CMOS power amplifiers by using CMOS transistors as the amplifying components. Input matching networks, gain stages, and output matching networks are common stages in the architecture of CMOS power amplifiers.

Power transfer is optimized by the input matching network's role in impedance matching between the amplifier and the stage before it. By doing this, it makes sure that the majority of the signal power is absorbed by the amplifier as opposed to being reflected back to the source. In addition to enhancing signal fidelity and reducing distortion, proper impedance matching.

The gain stages, which magnify the input signal to the appropriate output power level, are the heart of the power amplifier. They are made to offer high gain with a focus on linearity and low distortion. The inherent constraints of CMOS technology at high frequencies can make achieving high gain in CMOS power amplifiers difficult. Performance characteristics of the amplifier, such as power gain, bandwidth, and linearity, are influenced by parasitic capacitances and resistances in CMOS transistors.

Different design strategies are used to lessen these restrictions. The performance of the amplifier can be enhanced through careful device sizing, layout optimization, the use of inductive components, and matching networks. To attain the needed performance at high frequencies, strategies like cascode designs, impedance modification, and distributed amplifiers are used.

In order to maximize power transfer to the load, the output matching network guarantees impedance matching between the power amplifier and the load. It is essential for effectively supplying electricity and reducing reflections that can reduce the signal quality.

A thorough understanding of circuit theory, RF (Radio Frequency) design, and CMOS device characteristics is required while designing a CMOS power amplifier. The performance of the amplifier is modeled and examined using sophisticated simulation tools and optimization techniques, taking into consideration several factors like gain, power efficiency, linearity, and bandwidth.

CMOS power amplifiers can be designed for various frequency ranges, from low-frequency applications to mm-wave frequencies. The design considerations and techniques vary depending on the targeted frequency range and application requirements. High-frequency CMOS power amplifiers often require careful consideration of the parasitic elements, transmission line effects, and the use of advanced circuit topologies and techniques.

Ongoing research and advancements in CMOS power amplifier design aim to push the boundaries of performance. Researchers are exploring novel architectures, device structures, and circuit topologies to achieve higher power efficiency, wider bandwidth, improved linearity, and integration with other circuit blocks.

In conclusion, CMOS power amplifiers provide power amplification with good efficiency, compactness, and compatibility with typical CMOS processes, making them an attractive option for integrated circuit designs. They are essential parts of contemporary wireless communication systems because they permit faster data rates, greater energy economy, and more functionality.

\section{Existing Power Amplifier Design Techniques}
 CMOS technology possesses the ability to incorporate highly intricate digital circuitry, providing exceptional versatility and cost-effectiveness by integrating an entire radio system on a single chip. However, when it comes to power amplifier (PA) design, CMOS implementation poses significant challenges due to inherent limitations in standard CMOS processes from an RF perspective. These limitations include low oxide breakdown voltage, limited current drive capability, substrate coupling, and subpar quality and tolerance of on-chip passive components \cite{1320534, 1635249, 1465784}. These disadvantages have a detrimental impact on PA performance, particularly in terms of output power, efficiency, and linearity.
 
 High-data-rate and wide-bandwidth (BW) capabilities are required due to the growing need for wideband radio frequency (RF) transceivers in radar and satellite communication systems at sub-mm wave bands. At these higher frequencies, however, constructing wideband power amplifiers (PAs) with high power added efficiency (PAE) utilizing CMOS technologies is quite difficult \cite{8886521, 8354388, 9354447}.
Single broadband power amplifiers (PA) that can enable ultra-high data rate modulation and concurrently amplify multiple carriers are becoming more and more necessary in contemporary communication systems. The necessity for the PA to operate over a wider bandwidth than traditional narrowband amplifiers can be found in a number of applications, including imaging systems \cite{4977464}.

A popular and efficient method for achieving wideband features is the use of staggered power amplifiers (PAs) \cite{8678458, sapawl2012}. To achieve staggered wideband performance, this method includes designing the driver and main stages at two different center frequencies.

A low-power, full-band, low-noise amplifier for ultra-wideband receivers is described in study \cite{5497167}. For use in ultra-wideband applications, a low-power full-band low-noise amplifier (FB-LNA) is given. A MEMS tunable bandpass filter operating in the K band is described in Paper \cite{5728610}. It has a tiny dimension of 2.9 mm by 1.5 mm, a large continuous tuning range of 32\%, and a low insertion loss of -2.79-3.58 dB. It is suggested to use a straightforward CMOS broadband power amplifier architecture with great linearity and decent efficiency. To achieve a wider bandwidth from 0.9 to 3.5 GHz and low power consumption, the proposed power amplifier design used a staggered tuning technique composed of two stages of amplifier with distinct resonance frequencies \cite{sapawl2012}. An ultra-wideband (UWB) low-noise amplifier (LNA) employing a noise-cancelling approach and the TSMC 0.18-m RF CMOS technology is shown in paper \cite{shim2013}. In \cite{7202869}, a new method for designing transimpedance amplifiers (TIAs) that makes use of inverted transformer coils and stagger tuning is presented. In the paper \cite{7348667}, an inductorless 10 Gbps automatic gain control (AGC) circuit with speed-enhanced variable gain amplifiers (VGA), unique exponential function generators for extended linear performance in dB, power detectors and a comparator are presented. In paper \cite{8400821}, a T-type network and pole-tuning methods are used to study a broadband CMOS amplifier in the d band. a brand-new pole-tuning method using a T-type network to increase interstage bandwidth The stagger tuning technique is proposed in Paper \cite{8939351} at two separate frequencies, with an inductance of four stages selected to match the pole frequency at a lower frequency ($f_L$) and an inductance of the remaining four stages selected for the upper frequency ($f_H$). Citation \cite{9829838} states that the paper describes a two-stage CMOS 180-nm wideband power amplifier (PA). The PA uses defected-ground-structure (DGS) inductors and a superimposed staggered method. To produce a flat gain response throughout the full bandwidth, the design consists of a superimposed dual-band (SDB) driver stage and a wideband peaking main stage at the center frequency. By lowering insertion losses in the matching circuits, the usage of DGS inductors contributes to an improvement in PAE. With a chip area of 0.564 ${mm}^2$, the installed PA shows a power boost of 12 dB. It reaches a high fractional bandwidth (FBW) of 44.4\% at the center frequency and a saturated output power of 16.6 dBm.