\chapter{CONCLUSION}
In conclusion, the study looked into CMOS power amplifier design and optimization for improved gain and bandwidth in the  25GHz to 35GHz frequency range. Due to parasitic capacitances and frequency-dependent transconductance, the first single-stage power amplifier design had trouble matching input impedances and had lower gain. However, the cascaded power amplifier showed increased gain performance at a lower frequency by adding a second stage and using a staggered tuning strategy. In order to achieve better impedance matching, the design also included matching networks, which led to increased gain and decreased return loss. Gain, bandwidth, and matching performance of the power amplifier were successfully increased by the optimization efforts.

The proposed PA design impedance matched at two points: 27.12 GHz and 32.32 GHz compared to whole 25-35 GHz. That means the impedance hasn’t matched perfectly throughout the bandwidth. So, there is scope of designing the impedance matching network perfectly in future. Due to the presence of parasitic capacitance, there are some oscillations in the magnitude of $S_{21}$ parameter (power gain). We have to minimize the value of the parasitic capacitance of the power amplifier to reduce the oscillation in power gain curve.
\section{Future Work}
\begin{enumerate}
    

\item Explore Higher CMOS Technologies:
 The thesis paper focuses on designing CMOS power amplifiers using 90 nm CMOS technology. In the future, it would be beneficial to investigate the performance of power amplifiers using more advanced CMOS technologies, such as 65 nm, 45 nm, 28 nm or even smaller nodes. This would allow for higher integration levels, improved performance, and potentially reduced power consumption.

\item Investigate Different Circuit Topologies:
 The thesis paper may have focused on a specific circuit topology for power amplifiers. In the future, it would be interesting to explore different circuit architectures, such as Class F, Class G, or Class H, to determine their suitability for enhanced gain and bandwidth. Each circuit topology has its own advantages and limitations, so studying alternative options could provide valuable insights.

\item Implement Advanced Techniques for Bandwidth Enhancement:
 The thesis paper has employed staggered tuning technique using 2 stage power amplifier to enhance the bandwidth of the CMOS power amplifiers. Future work could involve investigating and implementing more advanced techniques like harmonic tuning, active inductors, or advanced matching networks to further improve the bandwidth performance.

\item Study Power Amplifier Linearity: 
The thesis paper may have primarily focused on gain and bandwidth enhancements. However, power amplifier linearity is another crucial aspect to consider, especially for applications where the amplifier needs to handle high-power signals. Future research could focus on studying linearity improvement techniques, such as linearization circuits or predistortion techniques, to achieve enhanced linearity while maintaining high gain and wide bandwidth.
\end{enumerate}
Overall, the research outcomes demonstrate the potential for enhancing CMOS power amplifier performance in wideband applications. By addressing the limitations and implementing design improvements, it is possible to achieve higher gain, wider bandwidth, and better impedance matching. The proposed recommendations pave the way for future research and development in this field, enabling the realization of high-performance CMOS power amplifiers for next-generation wideband communication systems.
